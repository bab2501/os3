\section{Experiment}
\paragraph{}
This experiment followed strict patterns to ensure that it would not affect the servers of each of the tested solutions. No real data was used in the tests. Each solution got an account created and fake data inserted into it.


\section{Requirments}
\paragraph{}
In this research we will be using five servers with the following specifications:

\begin{itemize}
	\item Intel Xeon processor
	\item 8GB ram
	\item SSD + HDD for storage
\end{itemize}

\paragraph{}
For the virtual server we will be using Xen 4.8.2 as hypervisor to implements the VMs. For the four physical servers we will be using each server to setup a separate functionality. Additional tools will be used, as well multiple logging and bench marking tools that allow us to keep a detailed log on the resource consumption and performance measurement.

\subsection{Testing environment}
\paragraph{}
To test the solutions the following tools were used to setup a testing environment:
\begin{itemize}
    \item Ubuntu 16.04.2
    \begin{itemize}
        \item Virtual Box - 5.0.40
        \begin{itemize}
            \item Windows 7 and 10 x64 - Fully updated
            \begin{itemize}
                \item Process Explorer - 16.21
                \item Windows 7/10 SDK - 7.1/10.0.15063.400
                \item Proxifier - 3.31
            \end{itemize}
        \end{itemize}
        \item Burp Suite - 1.7.23
        \item Apktool - 2.2.2
        \item Dex2jar - Commit dd9d722
        \item Jd\_Gui - Commit acd511f
    \end{itemize}
    \item Android - 7.1.2
    \begin{itemize}
        \item Proxydroid - 2.7.7
    \end{itemize}
\end{itemize}


\paragraph{}
Due to the cheer number of Password manager solutions we decided to test 4 solutions. Of these solutions 3 where chosen from the most "Popular" ones\cite{bestPasswordManagers}. The last one was choosen due to it being dutch\cite{Vaulteq}.
\begin{table}[H]
    \newcolumntype{s}{>{\columncolor[HTML]{AAACED}} p{1.8cm}}
    \centering
    \begin{tabular}{|s|c|c|c|c|}
        \hline
        \rowcolor[HTML]{AAACED} Password Managers: & Lastpass & Dashlane & Vaulteq & Keepass \\
        \hline
        Online & $\times$ & $\times$ & $\times$ &  \\
        \hline
        Offline & $\times$ & $\times$ & $\times$ & $\times$ \\
        \hline
        Windows & $\times$ & $\times$ & $\times$ & $\times$ \\
        \hline
        MAC & $\times$ & $\times$ & $\times$ & $\times$ \\
        \hline
        Linux & $\times$ &  & $\times$ & $\times$ \\
        \hline
        IOS & $\times$ & $\times$ & $\times$ & * \\
        \hline
        Android & $\times$ & $\times$ & $\times$ & * \\
        \hline
        \multicolumn{5}{|c|}{*Keepass for IOS/android only possible with forks}\\
        \hline
    \end{tabular}
    \caption{Password manager tested solutions}
    \label{tab:Scope_Password_Managers}
\end{table}
\clearpage
\subsection{Method}
\paragraph{}
For this project we decided to follow a methodology that starts from basic tests to increasingly complex ones. The steps we followed were:
\begin{itemize}
    \item ..
\end{itemize}
\subsubsection{Basic Input Validation}
\paragraph{}
We first decided to test the Input validation since its the core functionality of the program. As most programs it gets some user input, processes it and outputs the result as seen bellow on figure \ref{fig:PM_SimpleFunctionality}.
\begin{figure}[H]
    \centering
    \includegraphics[width=6cm]{Figures/PasswordManagerInputOutputSimple.png}
    \caption{Password manager's abstracted basic functionality}
    \label{fig:PM_SimpleFunctionality}
\end{figure}
\paragraph{}
In a password manager it is not different. For this test we imputed strings of varied lengths and with has many symbols as possible into all of the fields. Lastly, we tried to crash the program by imputing a very long string pattern of 20 thousand characters.
